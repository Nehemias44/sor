\documentclass[a4paper,12pt,titlepage]{article} % Tipo de plamtilla

% Paquetes
\usepackage[utf8]{inputenc}
\usepackage[spanish]{babel}
\usepackage{amsmath}
\usepackage{amssymb}
\usepackage{graphicx} % Para insertar imagenes
\usepackage[table,xcdraw]{xcolor} % Para interpretar colores
\usepackage{multirow} % Para las tablas
\usepackage{lastpage}
\usepackage{fancyhdr}
\usepackage{lipsum}  
\usepackage{multicol}
\usepackage{vmargin}
\setmargins{2cm}
{1cm}
{16.9cm}
{23.42cm}
{10pt}
{1cm}
{0pt}
{2cm}

% Colores
\definecolor{colorUngs}{HTML}{12719C}

% Variables
\newcommand{\titulo}{Las funciones del lenguaje de\par Jakobson}
\newcommand{\Autor}{Grupo 10}


\pagestyle{myheadings}
\pagestyle{fancy}
\fancyhf{}

\setlength{\headheight}{30pt}

\renewcommand{\headrulewidth}{1pt}
\renewcommand{\footrulewidth}{1pt}

\lhead{Universidad de General Sarmiento}
\rhead{Sistemas Operativos y Redes}

\lfoot{Camperoz, Gaona, XXX}
\rfoot{\thepage/\pageref{LastPage}}



\begin{document}
\begin{titlepage}
  \begin{center}
   \includegraphics[width=1.0\textwidth]{./Imagenes/logo_ungs.png}\par\vspace{0.1cm}
                  {\LARGE \bfseries Sistemas Operativos y Redes}\par\vspace{0.5cm}
                  {\huge \bfseries\textcolor{colorUngs}\titulo}\par\vspace{0.2cm}
                  {\large \bfseries \today}  
  \end{center}

\vspace{2.5cm}

  \bfseries{{\large \Autor}}
  \small\begin{itemize}
    \item Leandro Nehemias Camperoz
    \item Guillermo Gonzalo Gaona
    \item XXX
  \end{itemize}
  
  \vspace{1.5cm}

  \bfseries{\large Profesores}
  \small\begin{itemize}
    \item Mariano Vargas
    \item Noelia Sosa
    \item Ignacio Tula
  \end{itemize}
\end{titlepage}

{\Large \textbf{Resumen}}\newline

Segun Roman Jakobson la comunicacion es factible gracias a los siguientes factores y sus funciones del lenguaje.\newline

\begin{center}
 \begin{tabular}{|l|l|}
   \hline
  \textbf{Factor}& \textbf{Funcion} \\
  \hline
  Emisor& Emotiva o expresiva\\
  \hline
  Mensaje& Poetica\\
  \hline
  Receptor& Conativa\\
  \hline
  Contexto& Referencaial\\
  \hline
  Canal de comunicacion& Factor\\
  \hline
  Codigo comun& Metalinguistica\\
  \hline
\end{tabular} 
\end{center}

\begin{itemize}
  \item \textbf{Emotiva o expresiva}
    Se centra en el emisor, y permite expresar la nuestra actitud respecto a lo que estamos diciendo. Cuando hablamos no proporcionamos tan solo 
    informacion conceptual y objetiva, tambien incluimos elementos que tienen que ver con nuestra subjetividad.
  \item \textbf{Poetica} 
  \item \textbf{Conativa}
    Apunta al receptor dele mensaje. Es evidente en los inperativos y en los vocativos. Un inperativo se diferencia de una afirmacion en varios
    aspectos. No es ni verdadero ni falso, no se puede tranformarlo en una pregunta, es algo que se hace o no. 
  \item \textbf{Referencaial}
    Es la que enlaza el acto de comunicacion con su contexto o con su referente, es basica y casi nunca aparece aislada.
  \item \textbf{Fatica}

  \item \textbf{Metalinguistica} 
\end{itemize}

\end{document}
