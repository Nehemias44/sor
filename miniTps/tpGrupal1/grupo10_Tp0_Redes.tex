
\documentclass[a4paper,12pt,titlepage]{article} % Tipo de plamtilla

% Paquetes
\usepackage[utf8]{inputenc}
\usepackage[spanish]{babel}
\usepackage{amsmath}
\usepackage{amssymb}
\usepackage{graphicx} % Para insertar imagenes
\usepackage[table,xcdraw]{xcolor} % Para interpretar colores
\usepackage{multirow} % Para las tablas
\usepackage{lastpage}
\usepackage{fancyhdr}
\usepackage{lipsum}  
\usepackage{multicol}
\usepackage{vmargin}
\setmargins{2cm}
{1cm}
{16.9cm}
{23.42cm}
{10pt}
{1cm}
{0pt}
{2cm}

% Colores
\definecolor{colorUngs}{HTML}{12719C}

% Variables
\newcommand{\titulo}{Las funciones del lenguaje de\par Jakobson}
\newcommand{\Autor}{Grupo 10 Integrantes}


\pagestyle{myheadings}
\pagestyle{fancy}
\fancyhf{}

\setlength{\headheight}{30pt}

\renewcommand{\headrulewidth}{1pt}
\renewcommand{\footrulewidth}{1pt}

\lhead{Universidad de General Sarmiento}
\rhead{Sistemas Operativos y Redes}

\lfoot{Camperoz, Gaona, Morales}
\rfoot{ \thepage/\pageref{LastPage}}

\begin{document}
\begin{titlepage}
  \begin{center}
   \includegraphics[width=1.0\textwidth]{./Imagenes/logo_ungs.png}\par\vspace{0.1cm}
                  {\LARGE \bfseries Sistemas Operativos y Redes}\par\vspace{0.5cm}
                  {\huge \bfseries\textcolor{colorUngs}\titulo}\par\vspace{0.2cm}
                  {\large \bfseries \today}  
  \end{center}

\vspace{2.5cm}

  \bfseries{{\large \Autor}}
  \small\begin{itemize}
    \item Leandro Nehemias Camperoz
    \item Guillermo Gonzalo Gaona
    \item Federico Morales
  \end{itemize}
  
  \vspace{1.5cm}

  \bfseries{\large Profesores}
  \small\begin{itemize}
    \item Mariano Vargas
    \item Noelia Sosa
    \item Ignacio Tula
  \end{itemize}
\end{titlepage}

{\Large \textbf{Resumen}}\newline

Según Román Jakobson la comunicación es factible gracias a los siguientes factores y sus funciones 
en el lenguaje.\newline

\begin{center}
 \begin{tabular}{|l|l|}
   \hline
  \textbf{Factor}& \textbf{Función} \\
  \hline
  Emisor& Emotiva o expresiva\\
  \hline
  Mensaje& Poética\\
  \hline
  Receptor& Conativa\\
  \hline
  Contexto& Referencial\\
  \hline
  Canal de comunicación& Fatica\\
  \hline
  Código común& Metalingüística\\
  \hline
\end{tabular} 
\end{center}

\vspace{1cm}

\begin{itemize}
  \item \textbf{Emotiva o expresiva:}
    Se centra en el emisor y permite expresar nuestra actitud respecto a lo que estamos diciendo.
    Cuando hablamos no proporcionamos tan solo información conceptual y objetiva, también incluimos 
    elementos que tienen que ver con nuestra subjetividad.
  \item \textbf{Poética:}
    Se puede manifestar en la vida cotidiana pero de forma subordinada, es decir puede hacer 
    referencia a otras funciones.
  \item \textbf{Conativa:}
    Apunta al receptor del mensaje. Es evidente en los imperativos y en los vocativos.
    Un imperativo se diferencia de una afirmación en varios aspectos. 
    No es ni verdadero ni falso, no se puede transformar en una pregunta, es algo que se hace o no. 
  \item \textbf{Referencial:}
    Es la que enlaza el acto de comunicación con su contexto o con su referente, es básica y casi nunca aparece aislada.
  \item \textbf{Fatica:}
    Ciertos mensajes tienen que ver fundamentalmente con el canal de comunicación. Intentan mantenerlo
    abierto o comprobar que lo siga estando, incluso cerrarlo. 
  \item \textbf{Metalingüística:}
    Referente al propio lenguaje, por ejemplo, sirve para comprobar que compartimos el mismo código con el interlocutor.
\end{itemize}

\vspace{0.3cm}

{\Large \textbf{¿Cuales de estos conceptos se pueden relacionar con las redes de computadoras?}}\newline

En si el mismo concepto de comunicación se puede asociar al de la conexión, las computadoras forman una red conectándose entre si. Resulta claro que son necesarios medios para la conexión, estos varían dependiendo del tipo. No todas las conexiones son iguales, pueden ser como mínimo entre dos computadoras, entre miles o millones, a su vez que no todas las computadoras son del mismo tipo. Por lo tanto la comunicación entre ellas sera distinta, no es el mismo contexto cuando hay computadoras que hacen de intermediarios que cuando conectamos dos computadores directamente entre si. También es importante que estas se entiendan entre si, es decir que la información pueda ser interpretada por las partes sin que se produzcan errores.

\newpage

{\Large \textbf{Relación análoga entre conceptos}}\newline

Es posible establecer analogías respecto a los elementos que encontramos en las redes de computadoras y los conceptos explicados por Jakobson.

\begin{itemize}
\item \textbf{Contexto:} Podemos relacionarlo al tipo de red, no es lo mismo una red mínima entre dos computadoras, que la red privada de una empresa donde tenemos decenas o cientos de computadoras interconectadas.
\item \textbf{Emisor:} En redes de computadoras el rol del emisor puede ejercerlo por ejemplo un host cliente, el cual envía datos a un servidor atraves de internet.
\item \textbf{Medio/Canal:} Dependiendo del modelo de red tenemos distintos medios para conectarnos. Algunos ejemplos son el cable coaxial, cable trenzado(UTP), Fibra óptica, interfaces no cableadas (antenas wi-fi), etc.
\item \textbf{Receptor:} Siguiendo la situación descrita para dar una analogía de un emisor, el rol de quien recibe el mensaje lo puede cumplir el servidor, en otro caso en ves de enviar datos, el cliente (emisor) puede solicitar la visualización de un documento. El receptor, el servidor en este caso, puede aceptar o no esa solicitud.
\item \textbf{Código común:} Es imprescindible la existencia de pautas y registros en común entre las partes involucradas en la comunicación. Lo mismo ocurre en las redes de computadoras, la forma en que los datos son enviados deben respetar un protocolo, para que sea seguro que las computadoras entiendan la información recibida.     
\item \textbf{Mensaje:} Como vemos en los ítems anteriores, el mensaje puede ser un envió de datos, solicitudes para acceder a archivos, uso de aplicaciones web y muchos otros ejemplos mas, los cuales nombrarlos a todos es imposible.  
\end{itemize}

Luego de analizar los puntos anteriores, es evidente que todos ellos se relacionan constantemente entre si. 

\end{document}
